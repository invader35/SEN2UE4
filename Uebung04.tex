\documentclass[10pt,a4paper, parskip=half]{scrartcl}
\usepackage[utf8x]{inputenc}
\usepackage[german]{babel}
\usepackage{ucs}
\usepackage{amsmath}
\usepackage{amsfonts}
\usepackage{amssymb}
\usepackage{graphicx}
\usepackage{color}
\usepackage{listings}
\usepackage{hyperref}
\author{Schett Matthias}
\title{SEN-Übung 2.4}
\begin{document}
\definecolor{gray}{rgb}{0.9,0.9,0.9}
\lstset{language=[Visual]Basic, morekeywords={param, local}}
\maketitle

\section{Aufgabe 1}

\subsection{L\"{o}sungsidee}

Die Klasse WeatherStation soll eine Wetterstation abbilden.
\\Es gibt 3 Member: mName, mCelsius, mHumidity.
\\Weiters werden folgende Methoden verwendet:

\begin{enumerate}
	\item WeatherStation(std::string const \&name="", double celsius=0, double humidity=0);\label{Ctr1}
    \item std::string const \&GetName() const;\label{getname}
    \item void SetName(std::string const \&name);\label{setname}
    \item double GetCelsius() const;\label{getcels}
    \item void SetCelsius(double c);\label{setcels}
    \item double GetFahrenheit() const;\label{getfahren}
    \item void SetFahrenheit(double f);\label{setfahren}
    \item double GetHumidity() const;\label{gethum}
    \item void SetHumidity(double h);\label{sethum}
	\item void Print(std::ostream \&out) const;\label{print}
\end{enumerate}

ad \ref{Ctr1}
Der Konstruktor nimmt die Werte für die 3 Member an und weist sie mittels der Elementinialierliste zu.
Sollte nichts angegeben werden, gibt es Defaultwerte.

ad \ref{getname}
GetName() ist ein einfacher Getter und gibt mName zurück.

ad \ref{setname}
SetName() setzt mName auf name.

ad \ref{getcels}
GetCelsisus() ist ein einfacher Getter der mCelsius zurückgibt.

ad \ref{setcels}
SetCelsius() setzt mCelsius auf den Wert von c.

ad \ref{getfahren}
GetFahrenheit() gibt mCelsius in Fahrenheit umgerechnet zurück. Zur Berrechnung der Fahrenheit wird folgende Formel verwendet:
\[ T_F = T_C * 1.8 + 32 \]

ad \ref{setfahren}
SetFahrenheit() setzt mCelsius auf den in Celsisus umgerechneten Wert von f. Zur Berechnung wird folgende Formel verwendet:
\[ T_C = \frac{(T_F - 32)}{1.8} \]

ad \ref{gethum}
GetHumidity() gibt den Wert von mHumidity zurück.

ad \ref{sethum}
SetHumidity setzt den Wert von mHumidity auf den Wert von h.

ad \ref{print}
Print() kümmert sich um die Ausgabe. Diese gibt die Werte der drei Member auf dem Übergebenen std::ostream aus.

\subsection{Code}

Der Code befindet sich im Anhang unter \nameref{sec:WeatherStationHeader} und \nameref{sec:WeatherStationCpp}.
Der Testreiber ist im Anhang unter \nameref{sec:TestTreiberA1} zu finden.

\subsection{Testf\"{a}lle}

Der Testreiber testet die Funktionnalität indem er zuerst ein Objekt erstellt und anschließend sämtliche Methoden aufruft.
\paragraph{Ausgabe}

\lstinputlisting[breaklines=true, tabsize=4, showstringspaces=false,backgroundcolor=\color{gray}]{OutputA1.txt}

\newpage

\section{Aufgabe 2}

\subsection{Lösungsidee}

Die Klasse Weatherstations bildet eine Verwaltung für Wetterstationen ab, dazu gibt es die Member:
\\mMaxNumber (die maximale Anzahl an Wetterstationen), mNumberOfStations (die aktuelle Anzahl an Wetterstationen) und ein dynamischen Array
für Wetterstationen mStationArray.
\\Weiters gibt es noch folgende Funktionen:

\begin{enumerate}
	\item WeatherStation \& searchForColdest() const throw(WeatherException\&);\label{searchCold}
    \item WeatherStation \& searchForWarmest() const throw(WeatherException\&);\label{searchWarm}
    \item WeatherStations(size\_t maxNr=10);\label{Ctr}
	\item WeatherStations(WeatherStations const \&ws);\label{copyCtr}
	\item ~WeatherStations();\label{Destr}
	\item WeatherStations \&operator =(WeatherStations const \&ws);\label{asign}
	\item bool Add(WeatherStation const \&ws);\label{add}
	\item size\_t GetNrStations() const;\label{getNr}
    \item void PrintAll(std::ostream \&out) const;\label{printAll}
    \item void PrintColdest(std::ostream \&out) const;\label{printCold}
    \item void PrintWarmest(std::ostream \&out) const;\label{printWarm}
\end{enumerate}

ad \ref{searchCold}
Die Funktion searchForColdest() ist eine Private Funktion und sucht aus mStationArray die Wetterstation mit der niedrigsten Temperatur und gibt sie zurück.
Sollte mStationArray leer sein wird eine Exception geworfen um den Methodenaufruf abzubrechen.

ad \ref{searchWarm}
Die Funktion searchForWarmest() ist eine Private Funktion und sucht aus mStationArray die Wetterstation mit der höchsten Temperatur und gibt sie zurück.
Sollte mStationArray leer sein wird eine Exception geworfen um den Methodenaufruf abzubrechen.

ad \ref{Ctr}
Der Konstruktor weißt mMaxNumber maxNr zu und erstellt ein dynamisches Array mit der Anzahl von maxNr Einträgen, zusätzlich wird mNumberOfStations auf 0 gesetzt.
Sämtliche Operationen, werden mittels Elementinitalisierliste durchgeführt.

ad \ref{copyCtr}
Der Copy Konstruktor kopiert alle Werte von ws in das neue Objekt. Bei mStationArray wird eine tiefe Kopie (also das Kopieren der einzelnen Werte, nicht der Adressen) erstellt.

ad \ref{Destr}
Der Destruktor gibt den Speicher des dynamisch erstellten Arrays mittels delete [] wieder frei.

ad \ref{asign}
Der Zuweisungsoperator kopiert alle Werte von ws in das neue Objekt. Bei mStationArray wird eine flache Kopie (die Adresse wird kopiert) erstellt.

ad \ref{add}
Add() fügt eine neue Wetterstation in das mStationArray falls noch ein Platz frei sein sollte. Bei Erfolg gibt die Methode true zurück.

ad \ref{getNr}
GetNrStations() gibt mNumberOfStations zurück.

ad \ref{printAll}
PrintAll() ruft für jedes Element in mStationArray die Print() Funktion auf. Mittels out kan angegeben werden auf welchen Stream geschrieben werden soll.

ad \ref{printCold}
PrintColdest() ruft den Wert von searchColdest() auf dem Stream out aus.

ad \ref{printWarm}
PrintWarmest() ruft den Wert von searchWarmest() auf dem Stream out aus.

\subsection{Code}

Der Code befindet sich im Anhang unter \nameref{sec:WeatherManagerHeader} und \nameref{sec:WeatherManagerCpp}.
Der Testreiber befindet sich im Anhang unter \nameref{sec:TestTreiberA2}
Zusätzlich wird der Code aus Aufgabe 1 mitverwendet.

\newpage
\subsection{Testfälle}

Der Testreiber testet die Funktionnalität indem er zuerst ein Objekt erstellt, vier Wetterstationen versucht hinzuzfügen und anschließend sämtliche Methoden aufruft.
\paragraph{Ausgabe}

\lstinputlisting[breaklines=true, tabsize=4, showstringspaces=false,backgroundcolor=\color{gray}]{OutputA2.txt}

\newpage
\appendix

\section{Wetterstation Header}\label{sec:WeatherStationHeader}
\lstinputlisting[breaklines=true, tabsize=4, showstringspaces=false,backgroundcolor=\color{gray}, numbers=left]{WeatherStation/WeatherStation.h}

\section{Wetterstation Implementierung}\label{sec:WeatherStationCpp}
\lstinputlisting[breaklines=true, tabsize=4, showstringspaces=false,backgroundcolor=\color{gray}, numbers=left]{WeatherStation/WeatherStation.cpp}

\section{Testreiber A1}\label{sec:TestTreiberA1}
\lstinputlisting[breaklines=true, tabsize=4, showstringspaces=false,backgroundcolor=\color{gray}, numbers=left]{WeatherStation/Main.cpp}

\newpage
\section{Wetterstations Header}\label{sec:WeatherManagerHeader}
\lstinputlisting[breaklines=true, tabsize=4, showstringspaces=false,backgroundcolor=\color{gray}, numbers=left]{WeatherStationManager/WeatherStations.h}

\section{Wetterstations Implementierung}\label{sec:WeatherManagerCpp}
\lstinputlisting[breaklines=true, tabsize=4, showstringspaces=false,backgroundcolor=\color{gray}, numbers=left]{WeatherStationManager/WeatherStations.cpp}

\section{Testreiber A2}\label{sec:TestTreiberA2}
\lstinputlisting[breaklines=true, tabsize=4, showstringspaces=false,backgroundcolor=\color{gray}, numbers=left]{WeatherStationManager/Main.cpp}


\end{document}